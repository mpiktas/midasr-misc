\documentclass[nojss]{jss}
\usepackage{amsmath,amssymb}
\usepackage{bm}
\usepackage{multirow}
\usepackage{dcolumn}


\newcommand{\specialcell}[2][c]{%
  \begin{tabular}[#1]{@{}c@{}}#2\end{tabular}}

\author{Virmantas Kvedaras\\Vilnius University \And Vaidotas
  Zemlys\\Vilnius University}
\Plainauthor{Virmantas Kvedaras, Vaidotas Zemlys}

\title{Mixed Frequency Data Sampling Regression Models: the \proglang{R}
    Package \pkg{midasr}}
\Plaintitle{Mixed Frequency Data Sampling Regression Models: the R Package midasr}
\Shorttitle{R package \pkg{midasr}}

\Abstract{
  The implementation of MIDAS approach in the \proglang{R} package \pkg{midasr} is described.
}
\Keywords{MIDAS, specification test}
\Plainkeywords{MIDAS, specification test}
\Address{
  Vaidotas Zemlys\\
  Department of Econometric Analysis\\
  Faculty of Mathematics and Informatics\\
  Vilnius University\\
  Naugarduko g. 24, Vilnius, Lithuania\\
  E-mail:\email{Vaidotas.Zemlys@mif.vu.lt}\\
  URL: \url{http://vzemlys.wordpress.com}
}


%% need no \usepackage{Sweave.sty}

\begin{document}

\section{Introduction}

In econometric applications it is common to encounter time series
which are sampled in different frequencies, e.g. quarterly vs yearly,
etc. In order to use differently sampled series in regression analysis
one of the series is usually aggregated. It is evident that some of
the information is lost during such transformation. One of the
solutions to this problem is the mixed data sampling (MIDAS) approach introduced in
\cite{ghysels_touch_2002}. It has gained popularity in financial and some
macroeconomic applications (see e.g. \citealp{foroni:2012}, and
\citealp{sinko:2012}, for a recent overview of various
contributions). In the most cases, it is used for the forecasting purposes.

The main idea of MIDAS approach is based on observation that
aggregation of high frequency time series is actualy a specific
embedding of the high frequency domain to the low frequency
domain. Say we have yearly series $Y_t$ and quaterly series
$x_{\tau}$ and we want to estimate the model
\begin{align*}
  Y_{t}=f(x_{\tau})+\varepsilon_t
\end{align*}
The usual approach is to aggregate $x_{\tau}$ to a yearly sampling frequency:
\begin{align*}
  X_{t}=\frac{1}{4}(x_{4t}+x_{4t-1}+x_{4t-2}+x_{4t-3}),
\end{align*}
where we assume that yearly time series are observed at the same time
as the fourth quarter of the quarterly time series. Then we can
rewrite the model in the following form:
\begin{align*}
  Y_t=\alpha+\beta X_t+\varepsilon_t
\end{align*}
If we substitute the aggregation equation we see that this model is a
restricted form of the more general model:
\begin{align*}
  Y_t=\alpha+w_1x_{4t}+w_2x_{4t-1}+w_3x_{x4t-2}+w_4x_{4t-3}+\varepsilon_t
\end{align*}
We can extend this model for general frequency ratio $m$ and including
more lags resulting in so called U-MIDAS model:
\begin{align*}
  Y_t=\alpha+\sum_{h=0}^kw_hx_{tm-h}+\varepsilon_t.
\end{align*}
If frequency ratio $m$ is high, such
model might not be feasible to estimate, due to lack of degrees of
freedom. To solve this problem MIDAS approach suggests restricting the weights:
\begin{align*}
  w_h=g(h,\lambda), \quad, h=0,1,...,k
\end{align*}
where $g$ is some function and $\lambda$ is a vector of
hyper-parameters. In MIDAS literature function $g$ is usually chosen from
a fixed set of functions. The important question is then whether this
chosen function is the correct one. Recently Kvedaras and Zemlys
\cite{kz:2012} proposed a test which lets to test hypothesis whether the chosen weight function is appropriate.

Package \pkg{midasr} is aimed at applied researcher. It allows to
estimate the MIDAS regression model and test its feasibility.


\section{Theory}

\subsection{Simple MIDAS model}

Consider a situation where we observe two processes $y=\{y_t\in
\mathbb{R}, \ t=0,\pm 1,\pm 2\dots\}$  and $x=\{x_\tau\in \mathbb
  {R}, \ \tau=0, \pm 1, \pm2 \dots\}$ at different frequencies. For
each single low frequency observation $m$ high-frequency observations
are available. 

The MIDAS regression of $y$ on $x$ has the following representation
\begin{align} \label{eq:1}
  y_t=\sum_{h=0}^{d}\beta_hx_{tm-h}+\varepsilon_t. 
\end{align}

The coefficients $\beta_h$ are usually constrained by a functional constraint:
\begin{align*}
  \beta_h=g(\bm{\lambda},h), h=0,...,d
\end{align*}
where $\bm{\lambda}$ is a vector of hyper-parameters.


\section[Implementation in midasr package]{Implementation in \pkg{midasr} package}

\subsection{Data formats}

In order to estimate MIDAS regression high frequency regressor is
embedded into low frequency domain. This is done via \code{fmls}
function: 
\begin{Schunk}
\begin{Sinput}
> library(midasr)
> x <- 1:16
> fmls(x,3,4)
\end{Sinput}
\begin{Soutput}
     X.0/m X.1/m X.2/m X.3/m
[1,]     4     3     2     1
[2,]     8     7     6     5
[3,]    12    11    10     9
[4,]    16    15    14    13
\end{Soutput}
\end{Schunk}
It is assumed that for each low-frequency observation there are
exactly $m$ high-frequency observations.

\subsection{Model specification}

The specification of MIDAS regression is done via usual \code{formula}
interface with the help of the function \code{fmls}. For MIDAS
regression we need to specify the frequency ratio $m$, the number of
lags and the weight function. Since optimisation is used to estimate
the coefficients, the starting values must be supplied. 

\begin{Code}
midas_r(y~fmls(x,11,12,nealmon),start=list(x=c(0,0,0)))  
\end{Code}

The first argument of weight function must be the vector of
hyper-parameters. The second argument must be the number of the
coefficients the weight function returns. The number of the
coefficients is the number of lags plus one  so it is already defined
in formula specification. Accordingly it is passed to weight function
during optimisation. The additional arguments if there are any must be
listed into call to \code{fmls}.

Set up the data and load all necessary packages
\begin{Schunk}
\begin{Sinput}
> library(midasr)
> data("USunempr")
> data("USrealgdp")
> y <- diff(log(USrealgdp))
> x <- window(diff(USunempr), start = 1949)
> midas_r(y~fmls(x,11,12,nealmon),start=list(x=c(0,0,0)))
\end{Sinput}
\begin{Soutput}
MIDAS regression model
 model: y ~ fmls(x, 11, 12, nealmon) 
(Intercept)          x1          x2          x3 
    0.03243    -0.19137    15.62062     1.76868 

Function optim was used for fitting
\end{Soutput}
\end{Schunk}

\bibliography{midas}

\end{document}

