\documentclass[nojss]{jss}
\usepackage{amsmath,amssymb}
\usepackage{bm}
\usepackage{multirow}
\usepackage{dcolumn}


\newcommand{\specialcell}[2][c]{%
  \begin{tabular}[#1]{@{}c@{}}#2\end{tabular}}

\author{Virmantas Kvedaras\\Vilnius University \And Vaidotas
  Zemlys\\Vilnius University}

\title{Mixed Frequency Data Sampling Regression Models: the R Package \pkg{midasr}}
\Shorttitle{R package \pkg{midasr}}

\Abstract{
  The implementation of MIDAS approach in the \proglang{R} package \pkg{midasr} is described.
}
\Keywords{MIDAS, specification test}

\Address{
  Vaidotas Zemlys\\
  Department of Econometric Analysis\\
  Faculty of Mathematics and Informatics\\
  Vilnius University\\
  Naugarduko g. 24, Vilnius, Lithuania\\
  E-mail:\email{Vaidotas.Zemlys@mif.vu.lt}\\
  URL: \url{http://vzemlys.wordpress.com}
}


%% need no \usepackage{Sweave.sty}

\begin{document}

\section{Introduction}

The mixed data sampling (MIDAS) approach introduced in
\cite{ghysels_touch_2002} has gained popularity in financial and some
macroeconomic applications (see e.g. \citealp{foroni:2012}, and
\citealp{sinko:2012}, for a recent overview of various
contributions). In the most cases, it is used for the forecasting purposes.

\section{Theory}

\subsection{Simple MIDAS model}

Consider a situation where we observe two processes $y=\{y_t\in
\mathbb{R}, \ t=0,\pm 1,\pm 2\dots\}$  and $x=\{x_\tau\in \mathbb
  {R}, \ \tau=0, \pm 1, \pm2 \dots\}$ at different frequencies. For
each single low frequency observation $m$ high-frequency observations
are available. 

The MIDAS regression of $y$ on $x$ has the following representation
\begin{align} \label{eq:1}
  y_t=\sum_{h=0}^{d}\beta_hx_{tm-h}+\varepsilon_t. 
\end{align}

The coefficients $\beta_h$ are usually constrained by a functional constraint:
\begin{align*}
  \beta_h=g(\bm{\lambda},h), h=0,...,d
\end{align*}
where $\bm{\lambda}$ is a vector of hyper-parameters.


\section[Implementation in midasr package]{Implementation in \pkg{midasr} package}

\subsection{Data formats}

In order to estimate MIDAS regression high frequency regressor is
embedded into low frequency domain. This is done via \code{fmls}
function: 
\begin{Schunk}
\begin{Sinput}
> library(midasr)
> x <- 1:16
> fmls(x,3,4)
\end{Sinput}
\begin{Soutput}
     X.0/m X.1/m X.2/m X.3/m
[1,]     4     3     2     1
[2,]     8     7     6     5
[3,]    12    11    10     9
[4,]    16    15    14    13
\end{Soutput}
\end{Schunk}
It is assumed that for each low-frequency observation there are
exactly $m$ high-frequency observations.

\subsection{Model specification}

The specification of MIDAS regression is done via usual \code{formula}
interface with the help of the function \code{fmls}. For MIDAS
regression we need to specify the frequency ratio $m$, the number of
lags and the weight function. Since optimisation is used to estimate
the coefficients, the starting values must be supplied. 

\begin{Code}
midas_r(y~fmls(x,11,12,nealmon),start=list(x=c(0,0,0)))  
\end{Code}

The first argument of weight function must be the vector of
hyper-parameters. The second argument must be the number of the
coefficients the weight function returns. The number of the
coefficients is the number of lags plus one  so it is already defined
in formula specification. Accordingly it is passed to weight function
during optimisation. The additional arguments if there are any must be
listed into call to \code{fmls}.

Set up the data and load all necessary packages
\begin{Schunk}
\begin{Sinput}
> library(midasr)
> data("USunempr")
> data("USrealgdp")
> y <- diff(log(USrealgdp))
> x <- window(diff(USunempr), start = 1949)
> #midas_r(y~fmls(x,11,12,nealmon),start=list(x=c(0,0,0)))
\end{Sinput}
\end{Schunk}

%\bibliography{midas}

\end{document}

